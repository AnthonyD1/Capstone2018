\documentclass[11pt]{article}

\usepackage[margin=0.5in]{geometry}

\title{Block 8 Project Information \& Update}
\author{Anthony Delgado, Salem Ozaki, Joe Tortorello, and Jason Wang}
\date{18 April 2018}
\begin{document}

% What are you building?
% What do you hope to learn?
% Which tools (programming language, integrated development environment, database management system, MVC or testing framework, software for extracting and compiling documentation, etc.) have you chosen and why?

%In what kinds of work have you invested most of your effort.

%How did you divide your time among learning how to use your tools, design, coding, documentation, testing, and configuration? Can you show us examples of your code? )

\maketitle
\section*{What are we making?}
We are designing a web service to support a game of solitaire. This is not a product that end-users will utilize, but will be a service for front-end developers. A front-end developer can make a user-interface for any platform (e.g. web browser, mobile app, or even a desktop application), and report to our service each action that the user makes. The server will respond with the result of those actions.

\section*{Why a server-side application?}
Why would the front-end developer rely on our server-side application and not just make the entire game in the front-end code? There are two main reasons:
\begin{itemize}
	\item The user cannot modify the game logic or cheat when it is implemented on the server. This creates opportunities for having things like a universal high-score list that cannot be tampered with.
	\item Portability: the user can start the game on one device and continue it on another. This can even be across platforms (e.g. starting a game on a mobile device and continuing it on a desktop web browser).
\end{itemize}

\section*{Tools}
We are using the PHP programming language to implement this API. We are using the CakePHP web framework. CakePHP uses conventional naming and a scaffolding system to allow us to develop more rapidly.

Our application stores information in a database. During development we are using SQLite3, but CakePHP abstracts the database access, so our code that interacts with the database is not server-specific and could be changed to a more permanent server when preparing for production.

We are using PHPDoc to document our code and API. PHPDoc documentation is written in the same comment format familiar to users of JavaDoc and can output in HTML or several other formats.

We are all using JetBrains PHPStorm as an IDE, but have intentionally structured the project in such a way where it is not inherently linked to a particular development environment should others want to modify our project, or if our own tastes change.

\section*{Progress}
So far, most of our work has been dedicated to learning the CakePHP framework, and designing the API in such a way that it can take advantages of the facilities provided by the framework. As part of the learning process, we have developed a simple message-board application that uses a simple JSON API to post and receive messages.

\section*{Code}
The entire code base for the project can be found on GitHub. Particularly of interest may be the aforementioned message board demo. Find the controller code for that component at 
\end{document}
